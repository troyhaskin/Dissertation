\documentclass[12pt]{../UWMadThesis}


% =============================================================================================== %
%                                     Math Commands                                               %
% =============================================================================================== %


% ---------------------------------------------------------------------------- %
%                                Square Root Tail                              %
% ---------------------------------------------------------------------------- %
\DeclareRobustCommand{\NthRootInTeX}[2]{\root #1 \of {#2\:\!}}

\DeclareRobustCommand{\SquareRootCore}[2]{
    \setbox0=\hbox{\ensuremath{\NthRootInTeX{#1}{#2}}}
    \dimen0=\ht0
    \advance\dimen0-0.2\ht0
    \setbox2=\hbox{\vrule height\ht0 depth -\dimen0}
    {\box0\lower0.47pt\box2}
}

\DeclareRobustCommand{\Sqrt}[2][]{
    \mathchoice{\SquareRootCore{#1}{#2}}
               {\SquareRootCore{#1}{#2}}
               {\SquareRootCore{#1}{#2}}
               {\SquareRootCore{#1}{#2}}
}



% ---------------------------------------------------------------------------- %
%                              Derivative Commands                             %
% ---------------------------------------------------------------------------- %
\newcommand{\bigdiffn}[4]{\dfrac{#1{}^{#4}}{#1 #3{}^{#4}} \left[ #2 \right]}
\newcommand{\gendiffn}[4]{\dfrac{#1{}^{#4} #2}{#1 #3{}^{#4}}}

\newcommand{\diff}[3][d]{
    \ifthenelse{\equal{p}{#1}}{
        \gendiffn{\partial}{#2}{#3}{}
    }{
        \ifthenelse{\equal{b}{#1}}{
            \bigdiffn{d}{#2}{#3}{}
        }{
            \ifthenelse{\equal{bp}{#1}}{
                \bigdiffn{\partial}{#2}{#3}{}
            }{
                \gendiffn{d}{#2}{#3}{}
            }
        }
    }
}

\newcommand{\diffn}[4][d]{
    \ifthenelse{\equal{p}{#1}}{
        \gendiffn{\partial}{#2}{#3}{#4}
    }{
        \ifthenelse{\equal{b}{#1}}{
            \bigdiffn{#2}{#3}{#4}
        }{
            \ifthenelse{\equal{bp}{#1}}{
                \bigdiffn{\partial}{#2}{#3}{#4}
            }{
                \gendiffn{#1}{#2}{#3}{#4}
            }
        }
    }
}

\newcommand{\bigdiff}   [2] {\diff[b]{#1}{#2}}
\newcommand{\pdiff}     [2] {\diff[p]{#1}{#2}}
\newcommand{\bigpdiff}  [2] {\diff[bp]{#1}{#2}}
\let\frac\dfrac
\newcommand{\subs}      [2][]{\ensuremath{{}_{#1\text{\scriptsize #2}}}}
\newcommand{\sups}      [2][]{\ensuremath{{}^{#1\text{\scriptsize #2}}}}
\newcommand{\oneo}      [1]  {\ensuremath{\frac{1}{#1}}}




\newcommand{\Density}{\ensuremath{\rho}\xspace}
\newcommand{\Temperature}{\ensuremath{T}\xspace}
\newcommand{\Pressure}{\ensuremath{P}\xspace}
\newcommand{\IntEnergy}{\ensuremath{i}\xspace}
\newcommand{\Entropy}{\ensuremath{s}\xspace}
\newcommand{\Enthalpy}{\ensuremath{h}\xspace}
\newcommand{\ThCond}{\kappa}
\newcommand{\Viscosity}{\mu}
\newcommand{\DiffCoef}{\ensuremath{D}\xspace}

\newcommand{\isat}{\ensuremath{\IntEnergy\subs[\!]{sat}}\xspace}
\newcommand{\Psat}{\ensuremath{\Pressure\subs[\!\!]{sat}}\xspace}
\newcommand{\Tsat}{\ensuremath{\Temperature\subs[\!\!\:]{sat}}\xspace}
\newcommand{\SubL}{\subs[\!\!\:]{\rule{0pt}{8pt}$\textstyle\ell$}}
\newcommand{\SubG}{\subs[\!\!\:]{$\mathit{g}$}}

\newcommand{\rhol}{\ensuremath{\rho\SubL}\xspace}
\newcommand{\rhog}{\ensuremath{\rho\SubG}\xspace}
\newcommand{\il}{\ensuremath{i\SubL}\xspace}
\newcommand{\ig}{\ensuremath{i\SubG}\xspace}
\newcommand{\rhoul}{\ensuremath{\rhou\SubL}\xspace}
\newcommand{\rhoug}{\ensuremath{\rhou\SubG}\xspace}
\newcommand{\rhoil}{\ensuremath{\rhoi\SubL}\xspace}
\newcommand{\rhoig}{\ensuremath{\rhoi\SubG}\xspace}
\newcommand{\alphal}{\ensuremath{\alpha\SubL}\xspace}
\newcommand{\alphag}{\ensuremath{\alpha\SubG}\xspace}

\newcommand{\tauSat}{\ensuremath{\tau\subs[\!\!\:]{sat}}\xspace}
\newcommand{\deltaL}{\ensuremath{\delta\subs[\!\!\:]{\rule{0pt}{8pt}$\textstyle\ell$}}\xspace}
\newcommand{\deltaG}{\ensuremath{\delta\subs[\!\!\:]{$\mathit{g}$}}\xspace}

\newcommand{\rhoc}  {\ensuremath{\rho\subs{c}}\xspace}
\newcommand{\Tc}    {\ensuremath{T\subs{c}}\xspace}

\newcommand{\Skip}[1][0.45em]{\\[#1]}
\newcommand{\TCS}    {Thermodynamic Coexistence System\xspace}
\newcommand{\TCSRef} {\hyperref[Eqn:TCS]{\TCS}\xspace}
\newcommand{\MCS}    {Mechanical Coexistence System\xspace}
\newcommand{\MCSRef} {\hyperref[Eqn:MCS]{\MCS}\xspace}

\newcommand{\Afe}{\ensuremath{A\subs{\textsc{fe}}}}
\newcommand{\HFE}{Helmholtz free energy\xspace}
\newcommand{\EOS}{equation of state\xspace}

\newcommand{\Space}{\ensuremath{z}\xspace}
\newcommand{\Time}{\ensuremath{t}\xspace}
\newcommand{\Speeds}{\ensuremath{\mathbf{\lambda}}\xspace}

\DeclareMathOperator{\Ln}{Ln}
\DeclareMathOperator{\Abs}{Abs}
\DeclareMathOperator{\Inf}{Inf}
\DeclareMathOperator{\Exp}{Exp}
\DeclareMathOperator{\Rez}{R}

\let\originalleft\left
\let\originalright\right
\renewcommand{\left}{\mathopen{}\mathclose\bgroup\originalleft\;\!}
\def\left#1{\mathopen{}\mathclose\bgroup\originalleft#1\:\!}
\def\right#1{\aftergroup\egroup\:\!\originalright#1}


%\DefineNewLength{\RowSkip}{1.0em}
%\newcommand{\skp}[1][0.45em]{
%    \ifthenelse{\equal{#1}{}}{
%        \\[\RowSkip]
%    }{
%        \\[#1]
%    }
%}

\newcommand{\Del}[1][]{
    \partial_{#1}
}

\newcommand{\Vector}[1]{
    \underline{#1}
}

\newcommand{\Tensor}[1]{
    \underline{\underline{#1}}
}

\newcommand{\qConRaw}{\mathbf{q}}
\newcommand{\qCon}{\ensuremath{\qConRaw}\xspace}
\newcommand{\qPer}{\ensuremath{\widehat{\qConRaw}}\xspace}
\newcommand{\qSS} {\ensuremath{\qConRaw^0}\xspace}

\newcommand{\ConSys}{
    \Psi
}

\newcommand{\ConSysHEM}[1][HEM]{
    \ConSys_{\!\mbox{\tiny #1}}
}


\newcommand{\Flux}{
    \mathbf{F}
}
\newcommand{\Source}{
    \mathbf{S}
}

\newcommand{\Weight}{\beta}


\newcommand{\FluxFun}[2][]{
    \mathbf{F}_{#1}\left(#2\right)
}

\newcommand{\SourceFun}[2][]{
    \mathbf{S}_{#1}\left(#2\right)
}

\newcommand{\ResidualFun}[2][]{
    \mathbf{R}_{#1}\left(#2\right)
}

\newcommand{\Jacobian}[1][]{
    \mathbb{J}\subs{#1}
}

\newcommand{\JacobGen}[2]{
  \Jacobian[{\scriptscriptstyle #1}](#2)
}

\newcommand{\JacobF}{
    \Jacobian[F]
}


\newcommand{\JacobS}[1]{
    \JacobGen{S}{#1}
}

\newcommand{\FluxSS}{
    \mathbf{F}^{0}
}

\newcommand{\SourceSS}{
    \mathbf{S}^{0}
}

\newcommand{\JacobFSS}[1][\,\,\!]{
    \mathbf{J}_{\!{\scriptscriptstyle F}}^{0}{}#1
}

\newcommand{\JacobSSS}[1][\,\,\!]{
    \mathbf{J}_{\!{\scriptscriptstyle S}}^{0}#1
}

\newcommand{\BigO}[1]{
    \ensuremath{\mathcal{O}\!\left(#1\right)}
}


\newcommand{\Correl}[2]{
    f^{\mbox{\scriptsize cor}}_{#1}\left(#2\right)
}

\newcommand{\LpNorm}[2][2]{
    \ensuremath{\lvert\!\lvert#2\rvert\!\rvert_{#1}}
}

\newcommand{\Nudge}{
    \ensuremath{\!\!\;}
}

\newcommand{\hfg}{
    \ensuremath{h_{\mbox{\scriptsize fg}}}
}



%\NewEnviron{BoxedAlgorithm}[1][H]{
%    \begin{center}
%        \begin{minipage}{0.999\textwidth}
%            \centering
%            \fcolorbox{black}{white}{
%                \centering
%                \begin{minipage}[t]{0.85\textwidth}
%                    \begin{algorithm}[#1]
%                        \BODY
%                    \end{algorithm}
%                \end{minipage}
%            }
%        \end{minipage}
%    \end{center}
%}


\DeclareRobustCommand{\TH}  {thermal hydraulics\xspace}
\DeclareRobustCommand{\THc} {Thermal hydraulics\xspace}
\DeclareRobustCommand{\THcc}{Thermal Hydraulics\xspace}
\DeclareRobustCommand{\THs} {thermal hydraulic\xspace}

\DeclareRobustCommand{\CLaw}  {conservation law\xspace}
\DeclareRobustCommand{\CLaws} {conservation laws\xspace}


\newcommand{\rhou}{\ensuremath{\rho{u}}\xspace}
\newcommand{\rhoi}{\ensuremath{\rho{i}}\xspace}

\newcommand{\tr}{\ensuremath{{}\sups{\textsc{T}}}}
\newcommand{\mdotloss}[1][]{\ensuremath{\dot{m}'''\subs[\!\!\!\!\!#1]{loss}}\xspace}
\newcommand{\Keff}{\ensuremath{K\subs{eff}}}

\newcommand{\POfRhoRhoi}{\ensuremath{P\left(\rho,\frac{\rhoi}{\rho}\right)}}


\newcommand{\EqnSkip}[1][3em]{\ensuremath{\mbox{\rule{0.5em}{#1}}}\\}
\newcommand{\psiEOS}{\ensuremath{\psi}\subs{\textsc{eos}}}




%\DefineNewLength{\BarredLetterHeight}{0pt}
%\DefineNewLength{\BarredLetterWidth}{0pt}

%\newcommand{\eBB}{
%    \ensuremath{
%        \settoheight{\BarredLetterHeight}{e} % Height in current context
%        \settowidth{\BarredLetterWidth}{e}   % Width  in current context
%        e\mbox{\hspace{-0.57\BarredLetterWidth}\rule{0.035em}{0.96\BarredLetterHeight}} % bar
%    }
%}

%\newcommand{\TableSkip}{\rule[-1.4em]{0pt}{3.3em} \\[0pt]}
\definecolor{Gray}{gray}{0.93}


\newcommand{\LedineggCriterion}{$\tfrac{\partial\Delta{P}}{\partial(\rhou)}\bigr\rvert_{\text{int}} \le 
                                 \tfrac{\partial\Delta{P}}{\partial(\rhou)}\bigr\rvert_{\text{ext}}$}
                                
                                
\newcommand{\etal}{et al.\xspace}
\newcommand{\etc}{etc.\xspace}
\newcommand{\eg}{e.g.\xspace}
\newcommand{\ie}{i.e.\xspace}


\newcommand{\rhok}{ \ensuremath{\alpha\rho\subs{\phi}}\xspace}
\newcommand{\rhouk} {\ensuremath{\alpha\rhou\subs{\phi}}\xspace}
\newcommand{\rhoik} {\ensuremath{\alpha\rhoi\subs{\phi}}\xspace}
\newcommand{\alphak}{\ensuremath{\alpha\subs{\phi}\xspace}}
\newcommand{\uk}{\ensuremath{u\subs{\phi}}\xspace}
\newcommand{\ik}{\ensuremath{i\subs{\phi}}\xspace}
\newcommand{\CVvol}[1][k]{\ensuremath{\Omega_\text{#1}}\xspace}
\newcommand{\MCvol}[1][m]{\ensuremath{\Omega_\text{#1}}\xspace}
\newcommand{\CVsurf}[1][k]{\ensuremath{\Gamma_\text{#1}}\xspace}
\newcommand{\MCsurf}[1][m]{\ensuremath{\Gamma_\text{#1}}\xspace}








    \let\Oldalpha     \alpha     \renewcommand{\alpha}     {\ensuremath{\Oldalpha     }\xspace}
    \let\Oldbeta      \beta      \renewcommand{\beta}      {\ensuremath{\Oldbeta      }\xspace}
    \let\Oldgamma     \gamma     \renewcommand{\gamma}     {\ensuremath{\Oldgamma     }\xspace}
    \let\Olddelta     \delta     \renewcommand{\delta}     {\ensuremath{\Olddelta     }\xspace}
    \let\Oldepsilon   \epsilon   \renewcommand{\epsilon}   {\ensuremath{\Oldepsilon   }\xspace}
    \let\Oldvarepsilon\varepsilon\renewcommand{\varepsilon}{\ensuremath{\Oldvarepsilon}\xspace}
    \let\Oldzeta      \zeta      \renewcommand{\zeta}      {\ensuremath{\Oldzeta      }\xspace}
    \let\Oldeta       \eta       \renewcommand{\eta}       {\ensuremath{\Oldeta       }\xspace}
    \let\Oldtheta     \theta     \renewcommand{\theta}     {\ensuremath{\Oldtheta     }\xspace}
    \let\Oldvartheta  \vartheta  \renewcommand{\vartheta}  {\ensuremath{\Oldvartheta  }\xspace}
    \let\Oldkappa     \kappa     \renewcommand{\kappa}     {\ensuremath{\Oldkappa     }\xspace}
    \let\Oldlambda    \lambda    \renewcommand{\lambda}    {\ensuremath{\Oldlambda    }\xspace}
    \let\Oldmu        \mu        \renewcommand{\mu}        {\ensuremath{\Oldmu        }\xspace}
    \let\Oldnu        \nu        \renewcommand{\nu}        {\ensuremath{\Oldnu        }\xspace}
    \let\Oldxi        \xi        \renewcommand{\xi}        {\ensuremath{\Oldxi        }\xspace}
    \let\Oldpi        \pi        \renewcommand{\pi}        {\ensuremath{\Oldpi        }\xspace}
    \let\Oldvarpi     \varpi     \renewcommand{\varpi}     {\ensuremath{\Oldvarpi     }\xspace}
    \let\Oldrho       \rho       \renewcommand{\rho}       {\ensuremath{\Oldrho       }\xspace}
    \let\Oldvarrho    \varrho    \renewcommand{\varrho}    {\ensuremath{\Oldvarrho    }\xspace}
    \let\Oldsigma     \sigma     \renewcommand{\sigma}     {\ensuremath{\Oldsigma     }\xspace}
    \let\Oldvarsigma  \varsigma  \renewcommand{\varsigma}  {\ensuremath{\Oldvarsigma  }\xspace}
    \let\Oldtau       \tau       \renewcommand{\tau}       {\ensuremath{\Oldtau       }\xspace}
    \let\Oldupsilon   \upsilon   \renewcommand{\upsilon}   {\ensuremath{\Oldupsilon   }\xspace}
    \let\Oldphi       \phi       \renewcommand{\phi}       {\ensuremath{\Oldphi       }\xspace}
    \let\Oldvarphi    \varphi    \renewcommand{\varphi}    {\ensuremath{\Oldvarphi    }\xspace}
    \let\Oldchi       \chi       \renewcommand{\chi}       {\ensuremath{\Oldchi       }\xspace}
    \let\Oldpsi       \psi       \renewcommand{\psi}       {\ensuremath{\Oldpsi}\xspace}
    \let\Oldomega     \omega     \renewcommand{\omega}     {\ensuremath{\Oldomega     }\xspace}
    \let\OldGamma     \Gamma     \renewcommand{\Gamma}     {\ensuremath{\OldGamma     }\xspace}
    \let\OldLambda    \Lambda    \renewcommand{\Lambda}    {\ensuremath{\OldLambda    }\xspace}
    \let\OldSigma     \Sigma     \renewcommand{\Sigma}     {\ensuremath{\OldSigma     }\xspace}
    \let\OldPsi       \Psi       \renewcommand{\Psi}       {\ensuremath{\OldPsi       }\xspace}
    \let\OldDelta     \Delta     \renewcommand{\Delta}     {\ensuremath{\OldDelta     }\xspace}
    \let\OldXi        \Xi        \renewcommand{\Xi}        {\ensuremath{\OldXi        }\xspace}
    \let\OldUpsilon   \Upsilon   \renewcommand{\Upsilon}   {\ensuremath{\OldUpsilon   }\xspace}
    \let\OldOmega     \Omega     \renewcommand{\Omega}     {\ensuremath{\OldOmega     }\xspace}
    \let\OldTheta     \Theta     \renewcommand{\Theta}     {\ensuremath{\OldTheta     }\xspace}
    \let\OldPi        \Pi        \renewcommand{\Pi}        {\ensuremath{\OldPi        }\xspace}
    \let\OldPhi       \Phi       \renewcommand{\Phi}       {\ensuremath{\OldPhi       }\xspace}


\usepackage{enumitem}

%   Graphics path definition
\UWMadSetup{
    RelativeDirectory / {
        the-only-graphics-directory = Graphics
    }
}

%   Shim until merged back into the full document
\renewcommand{\Acronym}[1]{#1}

\ExplSyntaxOn

    \tl_set_eq:Nc \l_tmpa_tl {Gin@extensions}
    \clist_new:N \g__UWMad_RelativeDirectory_ImageExtensions_clist
    \clist_set:No
        \g__UWMad_RelativeDirectory_ImageExtensions_clist
        {\l_tmpa_tl}

    \clist_show:N \g__UWMad_RelativeDirectory_ImageExtensions_clist

\ExplSyntaxOff



\Institution{University of Wisconsin--Madison}


\begin{document}

\chapter{Introduction}

Stability of two-phase natural circulation systems is not a novel subject in and of itself.
However, this work aims to perform analysis on a novel geometry with unique characteristics using rigorous numerical methods.
Motivation for this effort is given through the examination of an experimental facility and preliminary data, both experimental and numerical.
Then, a literature overview of the field of stability analysis of fluid systems is presented with a broader perspective than that of this work.
Finally, the chapter concludes with the end goals of this investigation and an outline of the work.


\iffalse
\section{Reactor Cavity Cooling System}\label{Section:RCCS}
The primary motivation behind this stability work is the so-called \Acronym{RCCS}, which is a safety system for certain proposed Generation IV reactor designs.
A definition and discussion of this safety system for full-scale application is discussed first and followed by a description of an experimental test facility at the \TheUniversity.

\subsection{Full-Scale System}
The \Acronym{NGNP} is a thermal-spectrum, gas-cooled reactor designed to be able to produce electricity as well as process heat for industrial applications.
A novel feature of the \Acronym{NGNP} is the \Acronym{RCCS}.
The \Acronym{RCCS} is a natural circulation system of ducts designed as the \Acronym{NGNP}'s ultimate heat sink for decay heat under a number of accident scenarios.
There are two main designs under consideration which vary mostly in their working fluid: air-cooled and water-cooled.
This work will focus on the water-cooled \Acronym{RCCS} design, leaving air-cooled discussions left to the literature \cite{bechtelnationalinc._450_1993,generalatomics_gas_1996}.


\begin{figure}%
\centering
    \caption[ANL/UW-Madison water-cooled RCCS diagram]{   ANL/UW-Madison  water-cooled RCCS diagram.  
                Water flows from a tank, through the cold leg (blue), through the reactor cavity system, and through the hot leg (red) to some tank on the train at the same conditions (a closed circuit).}%
    \label{Figure:RCCSTotalSystem}%
    \includegraphics[scale=1.2]{RCCSTotalSystem.png}%
\end{figure}


A collaboration between \Acronym{ANL} and the \TheUniversity\ produced a base, full-scale \Acronym{RCCS} system consisting of a two independent piping networks with each network having four main lines/tanks.
\Cref{Figure:RCCSTotalSystem} outlines the basic design of system but precludes any detail near the reactor.
The elevation change of the system is on the order of several tens of meters with the reactor cavity portion being approximately twenty meters on its own.
Within the reactor cavity, the eight cold lines of the \Acronym{RCCS} splits into approximately 200 so-called riser pipes that line the cavity wall and ensconce the reactor (see \cref{Figure:RCCSMockup}).
The riser pipes then receive heat from the uninsulated reactor pressure vessel via convection and radiation.
Due to the heating, the fluid that fills the pipes is subjected to a buoyancy force that results in upward flow.

During steady-state operation of the reactor, there is a persistent, parasitic heat loss from the reactor vessel to the \Acronym{RCCS} of approximately $700$ kW.
If the reactor undergoes a loss-of-forced-flow accident with SCRAM, there is an expected peak decay heat load of $1.5$ MW on the \Acronym{RCCS}.
If there is no loss of onsite or offsite power, the \Acronym{RCCS} water storage tanks will be actively cooled, and the system is expected to maintain safe fuel temperatures for the duration of the accident.
In the event of loss of onsite and offsite power (so-called station blackout), the \Acronym{RCCS} can still continue to cool the reactor since the flow is naturally circulating, which is very important for the overall safety of the plant \cite[\SS{50.63}]{nuclearregulatorycomminission_us_2007}.

\begin{figure}%
\centering
    \caption[RCCS near-reactor-riser system]{   Cutaway picture of reactor vessel, RCCS ducts/pipes in the reactor cavity.  
                The low temperature fluid is in blue and the high temperaure fluid is in red.  
                The transparent boxes indicate network encasings. 
                Credit: Darius Lisowski, \TheUniversity.}%
    \label{Figure:RCCSMockup}%
    \includegraphics[scale=0.73]{RCCSMockup.png}%
\end{figure}

Without forced cooling, the reference temperature of the entire water system will continue to rise with continued decay heat exposure, and at some point during the transient, the water leaving the riser system will be at the saturation temperature of the system's gauge pressure.
However, the water will not immediately boil after leaving the due to the gravitational head of all the water above of the risers.
Rather, as the water flows to the top of the system, it will instantly boil as it passes into a region where the local pressure falls below the saturation pressure; a process called \textit{flashing}.
This sudden discontinuous jump in substance properties results in flow oscillations because the steam produced from the flash is approximately 1,000 times less dense than the liquid state on the cold side of the system.

\Cref{Figure:RCCSFullScaleMassFlowRate} shows a numerical simulation of the full-scale system.
Before flashing, the mass flow rate is non-oscillatory and increasing due to persistent heating with no cooling.
At the onset of flashing, the flow rate rapidly oscillates and evolves in a complicated manner.
At some point in the evolution, the system's flow rate stabilizes and evolves just as the single phase system.
The period, amplitude, and overall time evolution of these oscillations is subject to numerous factors and various linear and nonlinear effects.
The oscillations are commonly referred to as density wave instabilities since they are driven by the density differences across the system \cite{achard_analysis_1985}.
These incessant perturbations could potentially lead to large, erratic flow excursions that could pose a safety risk via extreme mechanical or thermal damage.

\begin{figure}%
    \centering
    \caption[RCCS mass flow rate under blackout conditions]{RCCS mass flow rate versus time under blackout conditions.}%
    \label{Figure:RCCSFullScaleMassFlowRate}%
    \includegraphics[width=0.9\textwidth]{PowerProfiles_MassFlowRateAnnotationsThesis.png}%
\end{figure}

\subsection{\TheUniversity\ RCCS Experiment}
Since the \Acronym{RCCS} discussed above exists purely on paper, an experiment was built at the \TheUniversity\ to directly observe and measure the behavior of an \Acronym{RCCS}-like system.
The experiment, presented in-brief by \cref{Figure:RCCSExperimentOverview}, is a scaled version of the full-scale \Acronym{RCCS} in terms of both dimension and number of risers.
While the full-scale system spans tens of meters with hundred of tubes, the experiment was scaled to a total height of approximately seven meters with three riser tubes.
Heaters mounted opposite the riser tubes provide the power to drive the natural circulation of the system.

\begin{figure}%
    \centering
    \caption[RCCS Experiment Full System diagram]{An overview of the whole RCCS experiment with important sections annotated.}%
    \label{Figure:RCCSExperimentOverview}%
    \includegraphics[height=0.50\textheight]{RCCSExperimentOverview.png}
    \vspace*{4em}
%\end{figure}%
%\begin{figure}%
    \centering
    \caption[RCCS Experiment Three Riser diagram]{An overview of the RCCS experiment's three riser/radiant heater setup.}%
    \label{Figure:RCCSExperimentHeatBox}%
    \includegraphics[width=0.80\textwidth]{RCCSExperimentHeatBox.png}%
\end{figure}

Due to the smaller scale and detailed design, a more precise numerical model could be made and benchmarked against data.
The full convection/radiation enclosure for the heater box was used to simulate station blackout conditions as in the full-scale with a scaled heat flux.
The mass flow rates from each of the risers for this accident simulation are shown in \cref{Figure:ExperimentMassFlowRateVsTime}.
The highly oscillatory nature of the flow rate after flashing begins is not only present but metastasized in this experiment.
Additionally, reverse flows in two of the risers is present (though the total system mass flow rate is always positive).
This local reversal of flow is more than likely systemic since short piping at the top allows a two-phase condition to exist at the top of the red riser.
While not expected in the full-scale system, this flow reversal is an interesting feature of the experiment's design and should be investigated.

\begin{figure}%
    \centering
    \begin{subfigure}[t]{\textwidth}
        \centering  
        \caption[ Mass flow rate versus time for the three riser tubes]{   
            Mass flow rate versus time for the three riser tubes of the RCCS experiment with station blackout accident conditions. 
            The colors of the plot lines correspond to the riser colors in \cref{Figure:RCCSExperimentRiserPlotAid}.}%
        \label{Figure:ExperimentMassFlowRateVsTime}%
        \includegraphics[width=0.75\textwidth]{ExperimentMassFlowRateVsTime.png}
    \end{subfigure}
    \vskip3em
    \begin{subfigure}[t]{\textwidth}
        \centering
        \caption[Basic layout of the RCCS experiment]{Basic layout of the RCCS experiment's numerical model.
                 The blue and red outlines denote the cooling and heating zones, respectively.}%
        \label{Figure:RCCSExperimentRiserPlotAid}%
        \includegraphics[height=0.35\textheight]{RCCSExperimentRiserPlotAid.png}%
    \end{subfigure}
\end{figure}

\newpage
\section{Literature}
Two-phase flow instabilities have been extensively studied across numerous industrial applications, including thermosiphons and power cycle loops; therefore, the literature of two-phase flow instabilities is extensive.
It should be noted that single phase instabilities do exist and are researched \cite{satoh_instability_1998}, but an in-depth overview of those mechanisms will not be given.
An overview of two-phase instabilities, classifications, and definitions is given below.
Then, specific discussion of efforts and techniques of analysis is discussed.

\input{Section2-Literature_Subsection1-InstabilityOverview}
\newpage
\subsection{Previous Analysis Efforts}
The two main types of analysis employed in stability analysis are linear and nonlinear, a superset of the former.
Every analysis begins with a defined set of the conservation/balance equations.
These equations possess all of the modeling information and assumptions in the analysis to be performed.
Three commonly used models are \cite{johnson_handbook_1998}:
\begin{itemize}
    \item{a homogeneous equilibrium model (HEM) where the distinct phases of a boiling fluid are treated as a single fluid with averaged properties;}
    \item{a separated flow model (SFM) where an additional momentum equation is added to the HEM allowing phase slip;}
    \item{a two-fluid model where the distinct phases are treated as separate partitions of a total volume, have completely separate properties, and only communicate through their shared interface.}
\end{itemize}
\Cref{Chapter:Theory} discusses the HEM and two-fluid equations, but for now, it suffices to say that these equations are nonlinear, coupled partial differentials equations that do not, in general, admit analytical solutions.

After the model has been decided, authors either linearize the equations or not and assess the system's stability through various techniques.
The specifics of the solution techniques are left to \cref{Section:StabilityTheory} since the techniques are all similar.
What makes the approaches more interesting is the models used, approximations made, and geometry considered.

Wallis and Heasley present one of the earliest efforts to analytically tackle two-phase flow oscillations \cite{wallis_oscillations_1961}.
They investigated a simple, closed natural circulation loop with pentane as the working fluid.
The analysis method looked at three sources of oscillations from a Lagrangian frame.
The first source was changes in riser buoyancy resulting from velocity perturbations and an equation for the marginal stability was derived in terms of the friction factor's derivative with respect to some steady-state velocity.
The second source was the heat input into the system with a theorized flow excursion for their loop.
Lastly, they investigated parallel channels but did not complete the analysis due to the then intractability of the solution.

Welander, though not a study of two-phase instabilities, investigates a simple, closed loop with point sources and proportional constitutive relations.\cite{welander_oscillatory_1967}.
While the treatment is similar to Wallis and Heasley's, Welander's more mathematical approach is more in-line with this work's goals.
Additionally, Welander derived an asymptotic steady-state solution and also compared the analytical neutral boundary with numerical experiments to confirm the boundary's validity.
Zvirin and Greif continued this work by examining how an arbitrary initial condition evolved toward Welander's steady state \cite{zvirin_transient_1979}.
They found that while the solution approached the analytical steady-state, the solution was without oscillation and concluded that the oscillation characteristics were strongly dependent on the shape of the initial condition.

Achard \etal investigated material wave oscillations in a horizontal boiling channel using both linear and nonlinear analysis \cite{achard_analysis_1985}.
They derived a lumped parameter integrodifferential equation.
Upon linearizing the equation and using the friction and sub-cooling numbers as degrees of freedom, they found two absolutely unstable regions, several conditionally unstable regions, and one absolutely stable region.
The conditionally unstable regions were found to depend on how actual system's state evolved in time and moved through the stability space.
They also performed a nonlinear analysis where the parameter values to maintain stability were obtained by nonlinearly solving the lumped parameter equation for a given perturbation amplitude.
The nonlinear analysis showed that stable parameter curves existed for their system but the absolutely stable region shrank with increasing perturbation amplitude.

Lee and Ishii performed a linear analysis similar to Welander, but the system was explicitly two-phase, had a quadratic frictional dependence in velocity with a constant friction factor, and had a distributed heat load \cite{sangyonglee_thermally_1990}.
Additionally, as an extension of the Zvirin and Greif's methodology, Lee and Ishii divided the loop into size different regions with average properties and solved for the stability boundary using linear analysis.

Lee and Lee performed nearly the same analysis as Lee and Ishii with the added difficulty of a variable, flow-regime dependent friction factor \cite{lee_linear_1991}.
Guanghui \etal also performed a similar analysis over a simple, closed loop that compared very well experiments \cite{guanghui_theoretical_2002}.

Knaani and Zvirin investigated the existence of multiple steady-states for a simple, closed loop undergoing boiling\cite{knaani_bifurcation_1993}.
The analysis was done with HEM and incorporated a guess and check procedure for finding a solution to the nonlinear problem.
They ultimately found multiple steady-states for the closed system due to the non-monotonic nature of the buoyancy term during phase change.
Nayak \etal also performed analysis on a simple, closed loop but used a four equation model (two for mass and mixture equations for energy and momentum) and several different two-phase friction multipliers \cite{nayak_study_2007}.

Lee and Pan undertook an \Acronym{HEM} for a two-phase natural circulation system with two parallel, heated channels \cite{lee_nonlinear_2005}.
This treatment is the first work in this review to explicitly analyze a non-simple closed loop.
Through a number of piece-wise integrations along the loop, the authors arrived at a set of ordinary differential equations that satisfied the zero pressure gradient requirement of the closed loop.
Using the channel inlet sub-cooling as a parametric and a blackbox solver, the authors compared the steady-state channel flow rates with experimental data.
The authors then found a stability curve for even heating of the channels that possessed two unstable regions and one stable region.
The author concluded with a parametric study of how the stability region shifted with the addition of an orifice plate.










\section{Research Purpose}\label{Section:Purpose}

The purpose of this research is to theoretically investigate the linear stability of a closed loop, natural circulation system undergoing phase-change with water as a working fluid.
The goal is to ultimately apply stability theory to three parallel channels in a closed circuit with mass loss.
While the geometry may be a simple extension of other works, a defining characteristic of this system is that of mass loss.
As the system boils, the inventory is draining and a question that has not been completely answered is how does this affect the stability and overall evolution of the system.
In addition to the geometrical considerations, this work also explores a novel discretization scheme to the standard set of thermohydraulic equations that aims to more accurately represent branching flows than standard channel flow models.
Also, this work applies a full non-linear solver to the set of discretized equations with no operator splitting over a time-step.
Lastly, the linear stability of the system over a transient is determined through a detailed volume integration of the governing equations and an eigenvalue calculation.


The outline of the topics to be discussed in this work is:
\begin{enumerate}[topsep=0pt,parsep=0pt,itemsep=2pt]
    \item{conservation laws and pertinent thermohydraulic equations to solve;}
    \item{the discretization scheme and approximations applied to the equations;}
    \item{stability theory as applied to the set of nonlinear equations;}
    \item{the nonlinear solver used;}
    \item{results of transient simulations and associated stability calculations;}
    \item{conclusions from the presented results and future work.}
\end{enumerate}
Each section builds upon the last and works to fulfill the delineated goals above.



\bibliography{../Sources}
\bibliographystyle{unsrt}

\fi

\end{document}




