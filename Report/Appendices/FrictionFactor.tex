\appendix{Friction Factors}
\label{Appendix:FrictionFactors}

As mentioned in \cref{SubSubSection:StressTensor}, the mono-directional bulk momentum stress tensor is taken to have the algebraic form
\begin{equation}
    \tau_{ij} z_i n_j = \Delta{P}_{\text{fric}} = \frac{1}{2}
        f\subs{\textsc{d}}\,\frac{L\subs{char}}{D\subs{eff}}\,\rho\,v^2,
\end{equation}
where $f\subs{\textsc{d}}$ is the Darcy friction factor, $L\subs{char}$ is a characteristic (path) length, $D\subs{eff}$ is an effective flow diameter, $\rho$ is the fluid density, and $v$ is the fluid velocity.
For correctness, the method used to calculate the friction factor must account for different kinematic flow regimes and distinguish between single and two-phase flow.
The following appendix outlines the methodology used in this work.

For slow or very viscous flows, the friction factor is found to be a function of its Reynolds number:
\begin{equation}
    \text{Re} = \frac{\rho\,v\,D\subs{eff}}{\mu}
\end{equation}
where $\mu$ is the dynamic viscosity.
This is primarily due to the layered and ordered nature of the flow that results in much of the fluid's traction coming from its inherent viscosity \cite{nellis_heat_2009}.
The Darcy friction approximation used for Reynolds numbers at or below \num{2000}, called \textit{laminar flow}, is 
\begin{equation}
    f\subs{\textsc{d}} = \frac{64}{\text{Re}}.
\end{equation}
For flows with Reynolds numbers above \num{4000}, called \textit{turbulent flow}, the Darcy friction factor is well-approximated by the Colebrook-White equation
\begin{equation}
    \frac{1}{\Sqrt{f\subs{\textsc{d}}}} = -2 \Log_{10}\left(\frac{\varepsilon\subs{rel}}{3.7} + \frac{2.51}{\text{Re}\;\;\Sqrt{f\subs{\textsc{d}}}}\right)
\end{equation}
where $\varepsilon\subs{rel}$ is the relative surface roughness and $\text{Re}$ is the Reynolds number.
The relative surface roughness is a material-dependent quantity determined by measuring how smooth the finished inner surface is and dividing that measure by the channel's hydraulic diameter.
A constant value of \num{1E-4} was used in this work.
Despite its seemingly arbitrary nature, the Colebrook-White equation is derived from physical arguments in turbulent flow theory \cite{matthew_colebrook-white_1990}; however, to avoid yet another level of implicit solution in this work, an explicit approximation to the implicit Colebrook-White equation due to Serghides is used
\begin{equation}
    \frac{1}{\Sqrt{f\subs{\textsc{d}}}} = f_a - \frac{(f_b - f_a)^2}{f_c - 2 f_b + f_a}
\end{equation}
where
\begin{subequations}
    \begin{align}
        f_a &= -2 \Log_{10}\left( \frac{12}       {\text{Re}} + \frac{\varepsilon\subs{rel}}{3.7}\right) \\[0.5em]
        f_b &= -2 \Log_{10}\left( \frac{2.51\,f_a}{\text{Re}} + \frac{\varepsilon\subs{rel}}{3.7}\right)\\[0.5em]
        f_c &= -2 \Log_{10}\left( \frac{2.51\,f_b}{\text{Re}} + \frac{\varepsilon\subs{rel}}{3.7}\right)
    \end{align}
\end{subequations}
The method is derived by applying Steffensen's Method to the Colebrook-White equation for a particular initial guess value, and the additional arithmetic overhead is balanced by its explicit nature and excellent comparison to Colebrook-White over a large sample space as detailed in \cite{burden_numerical_2004,winning_explicit_2012}.
For flows between the upper and lower Reynolds number bounds of \num{2000} and \num{4000}, the so-called \textit{transition region}, a log-linear affine homotopy similar to the mixture property calculation was used
\begin{equation}
    f_{\text{\textsc{d},tran}} =    (1-w_{\text{\textsc{d}}}) f_{\text{\textsc{d}}}(\text{Re} = \num{2000}) + 
                                       w_{\text{\textsc{d}}}  f_{\text{\textsc{d}}}(\text{Re} = \num{4000})
\end{equation}
where the weight function is 
\begin{equation}
    w_{\text{\textsc{d}}} = \frac{\Log_{10}(\text{Re}) - \Log_{10}(\text{2000})}{\Log_{10}(\text{4000}) - \Log_{10}(\text{2000})}.
\end{equation}

The above functional forms and correlations are all based upon the presence of a single phase.
There are a number of methods used to account for the increased frictional pressure drop due to the existence of another phase; the particular one chosen here is the two-phase multiplier.
That is, the two-phase pressure drop is expressed as a scalar multiple of the saturated liquid drop:
\begin{equation}
    \Delta{P}_{\text{fric,two}} = \alpha_{\text{two}}\,\Delta{P}_{\text{fric,}\ell}
\end{equation}
where $\Delta{P}_{\text{fric,}\ell}$ is calculated per above and the two-phase multiplier $\alpha_{\text{two}}$ is calculated using the model proposed by Friedel:
\begin{equation}
    \alpha_{\text{two}} = \beta_1 + \frac{3.24 \beta_2 \beta_3}{\text{Fr}^{0.045}\,\text{We}^{0.035}}
\end{equation}
where 
\begin{align*}
    \beta_1   &= (1-x)^2 + x^2 \left( \frac{\rhol}{\rhog}\,\frac{f_{\text{\textsc{d,}}\mathit{g}}}{f_{\text{\textsc{d,}}\ell}} \right) \\[1em]
    \beta_2   &= x^{0.78} (1-x)^{0.24} \\[1em]
    \beta_3   &= \left(\frac{\rhol}{\rhog}\right)^{0.91}\,\left(\frac{\mu\SubG}{\mu\SubL}\right)^{0.19}\,\left(1 -\frac{\mu\SubG}{\mu\SubL}\right)^{0.7} \\[1em]
    \text{Fr} &= \frac{v^2}{g\,D_{\text{eff}}} \\[1em]
    \text{We} &= \frac{\rho\,v^2\,D_{\text{eff}}}{\sigma}
\end{align*}
where $x$ is the quality and $\sigma$ is the surface tension \cite{collier_convective_1994}.
The dimensionless quantities $\text{Fr}$ and $\text{We}$ are the Froude number and Weber number, respectively.
As may seem evident, this correlation is extremely empirical but is the recommended correlation to use for the set of states expected to be seen in this work.



