\appendix{Total Derivatives}\label[appendix]{Appendix:TotalDerivatives}

In \cref{Chapter:Theory}, a function for the thermodynamic pressure was presented that was explicitly a function of the density $\rho$ and the conservative internal energy \IntEnergy/$\rho$.
For that function, the total derivative with respect to the conserved variable $\rho$ is

\begin{align}
    \deriv{P}{\rho} &= \pderiv{P}{\rho} - \frac{\rhoi}{\rho^2}\pderiv{P}{i} \\[0.4em]
                   &= \pderiv{P}{\rho} - \frac{i}{\rho}\pderiv{P}{i} 
\end{align}

and with respect to the conserved variable \rhoi is
\begin{equation}
    \deriv{P}{(\rhoi)} = \oneo{\rho}\pderiv{P}{i}
\end{equation}


However, since the \EOS presented in \cref{Section:EOS} is explicitly a function of density and temperature, the derivatives that arise from Jacobians have a more complicated form.
The relationship between a thermodynamic function $\psi$($\rho$,\IntEnergy) and \psiEOS($\rho$,\Temperature) is, in conservative form,
\begin{equation}
    \psi\left(\rho,\frac{\rhoi}{\rho}\right) = \psiEOS\left[\rho,T\left(\rho,\frac{\rhoi}{\rho}\right)\right].
\label{Eqn:PressureFunctionRelation}
\end{equation}
From this equation, the total derivative with respect to $\rho$ is
\begin{equation}
    \deriv{\psi}{\rho} = \pderiv{\psiEOS}{\rho} + 
                        \pderiv{\psiEOS}{T}    
                        \left[\pderiv{T}{\rho}\biggr)_{\!\!\!\! i} - \frac{i}{\rho}\pderiv{T}{i}\biggr)_{\!\!\!\!    \rho}\right]
    \label{Eqn:TotalwrtRho}
\end{equation}
and  with respect to \rhoi is
\begin{equation}
    \deriv{\psi}{(\rhoi)} = \oneo{\rho}\pderiv{\psiEOS}{T} \pderiv{T}{i}\biggr)_{\!\!\!\!    \rho}
    \label{Eqn:TotalwrtRhoi}
\end{equation}
where the constancy of properties has been emphasized with purpose.
The derivative of the temperature with respect to density in \cref{Eqn:TotalwrtRho} seems unwieldy since the IAPWS takes those properites to be independent; however, the requirement that the derivative is taken along a line of constant internal energy admits a solution.
By the Implicit function theorem, the derivative of temperature with respect to density holding some thermodynamic property $\psi$ constant is
\begin{equation}
    \pderiv{T}{\rho}\biggr)_{\!\!\!\! \psi} = - \pderiv{\psi}{\rho}\biggr)_{\!\!\!\! T} \left[\pderiv{\psi}{T}\biggr)_{\!\!\!\! \rho} \right]^{-1}
    \label{Eqn:dTdrho}
\end{equation}
provided that $\partial\psi/\partial{T}$ is non-zero and $\psi(\rho,\Temperature)$ has (at least) a continuous first derivative at the evaluation state \cite{greenberg_advanced_1998}.
By the Inverse function theorem, the derivative of temperature with respect to some thermodynamic property $\psi$ is
\begin{equation}
    \pderiv{T}{\psi}\biggr)_{\!\!\!\!    \rho} = \left[\pderiv{\psi}{T}\biggr)_{\!\!\!\!    \rho} \right]^{-1}.
    \label{Eqn:dTdi}
\end{equation}
provided that $\partial\psi/\partial{T}$ is non-zero and $\psi(\rho,\Temperature)$ has (at least) a continuous first derivative at the evaluation state \cite{greenberg_advanced_1998}.
\Cref{Eqn:dTdi} is useful since $\partial\psi/\partial{T}$ is directly calculable from the IAWPS-95 \EOS in the single phase region (two-phase conditions require a numeric approximation to the derivative).

For the range of states considered in this work, the requirement that all the properties vary with temperature (i.e., non-zero derivative) is easily found to be valid.
Concerning the smoothness condition, while the IAPWS-95 implementation is by-construction smooth in the single phase region, it is \emph{not} continuously differentiable at the two-phase boundary (only continuous) with either quality or void fraction weighting.  
Therefore, with a homogeneous equilibrium flow model, the inverse and implicit function theorems are inapplicable without some exception used at the boundary.
That exception has been made in the implementation: for a purely saturated state, the derivatives used are those of the single phase fluid.

Lastly, it will be noted that while conservation of density through $\rho$ is universal, conservation of energy through \rhoi is an approximation.
The approximation of conservation through \rhoi ignores kinetic and potential energy of the fluid.
While potential energy is usually included as a source term only, kinetic energy is commonly added to the internal energy to form the total energy of the system.
If the energy equation was required to explicitly conserve thermal and kinetic energy, the conserved property would be defined as 
\begin{equation}
    \eBB = \rhoi + \oneo{2} \frac{\rhou^2}{\rho}
\end{equation}
and internal energy would have the form
\begin{equation}
    i = \frac{\eBB}{\rho} - \oneo{2}\left(\frac{\rhou}{\rho}\right)^2.
\end{equation}
Thus, in this formulation, the thermodynamic derivatives would have a momentum dependence in any Jacobian containing them.
However, the current work does not explicitly conserve kinetic energy, and exploration of this form is left to future work.

