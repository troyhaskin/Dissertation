\documentclass{UWMadThesis}

% ============================================================ %
%                      Thesis Information                      %
% ============================================================ %
\Title{On the Stability of Natural Circulation Loops with Phase Change}
\Author{Troy C. Haskin}
\Program{Nuclear Engineering and Engineering Physics}
\Doctorate{}
\Dissertation{}

\DefenseDate{September 20, 2016}
\University{University of Wisconsin--Madison}

\Advisor{Michael L. Corradini}       {Professor}{Engineering Physics}
\CommitteeMember{Robert J. Witt}     {Professor}{Engineering Physics}
\CommitteeMember{Jake Blanchard}     {Professor}{Engineering Physics}
\CommitteeMember{Gregory A. Moses}   {Professor}{Engineering Physics}
\CommitteeMember{Christopher Rutland}{Professor}{Mechanical Engineering}



% ================================================================= %
%                        Document class options                     %
% ================================================================= %

% =============================================================================================== %
%                                     Math Commands                                               %
% =============================================================================================== %


% ---------------------------------------------------------------------------- %
%                                Square Root Tail                              %
% ---------------------------------------------------------------------------- %
\DeclareRobustCommand{\NthRootInTeX}[2]{\root #1 \of {#2\:\!}}

\DeclareRobustCommand{\SquareRootCore}[2]{
    \setbox0=\hbox{\ensuremath{\NthRootInTeX{#1}{#2}}}
    \dimen0=\ht0
    \advance\dimen0-0.2\ht0
    \setbox2=\hbox{\vrule height\ht0 depth -\dimen0}
    {\box0\lower0.47pt\box2}
}

\DeclareRobustCommand{\Sqrt}[2][]{
    \mathchoice{\SquareRootCore{#1}{#2}}
               {\SquareRootCore{#1}{#2}}
               {\SquareRootCore{#1}{#2}}
               {\SquareRootCore{#1}{#2}}
}



% ---------------------------------------------------------------------------- %
%                              Derivative Commands                             %
% ---------------------------------------------------------------------------- %
\newcommand{\bigdiffn}[4]{\dfrac{#1{}^{#4}}{#1 #3{}^{#4}} \left[ #2 \right]}
\newcommand{\gendiffn}[4]{\dfrac{#1{}^{#4} #2}{#1 #3{}^{#4}}}

\newcommand{\diff}[3][d]{
    \ifthenelse{\equal{p}{#1}}{
        \gendiffn{\partial}{#2}{#3}{}
    }{
        \ifthenelse{\equal{b}{#1}}{
            \bigdiffn{d}{#2}{#3}{}
        }{
            \ifthenelse{\equal{bp}{#1}}{
                \bigdiffn{\partial}{#2}{#3}{}
            }{
                \gendiffn{d}{#2}{#3}{}
            }
        }
    }
}

\newcommand{\diffn}[4][d]{
    \ifthenelse{\equal{p}{#1}}{
        \gendiffn{\partial}{#2}{#3}{#4}
    }{
        \ifthenelse{\equal{b}{#1}}{
            \bigdiffn{#2}{#3}{#4}
        }{
            \ifthenelse{\equal{bp}{#1}}{
                \bigdiffn{\partial}{#2}{#3}{#4}
            }{
                \gendiffn{#1}{#2}{#3}{#4}
            }
        }
    }
}

\newcommand{\bigdiff}   [2] {\diff[b]{#1}{#2}}
\newcommand{\pdiff}     [2] {\diff[p]{#1}{#2}}
\newcommand{\bigpdiff}  [2] {\diff[bp]{#1}{#2}}
\let\frac\dfrac
\newcommand{\subs}      [2][]{\ensuremath{{}_{#1\text{\scriptsize #2}}}}
\newcommand{\sups}      [2][]{\ensuremath{{}^{#1\text{\scriptsize #2}}}}
\newcommand{\oneo}      [1]  {\ensuremath{\frac{1}{#1}}}




\newcommand{\Density}{\ensuremath{\rho}\xspace}
\newcommand{\Temperature}{\ensuremath{T}\xspace}
\newcommand{\Pressure}{\ensuremath{P}\xspace}
\newcommand{\IntEnergy}{\ensuremath{i}\xspace}
\newcommand{\Entropy}{\ensuremath{s}\xspace}
\newcommand{\Enthalpy}{\ensuremath{h}\xspace}
\newcommand{\ThCond}{\kappa}
\newcommand{\Viscosity}{\mu}
\newcommand{\DiffCoef}{\ensuremath{D}\xspace}

\newcommand{\isat}{\ensuremath{\IntEnergy\subs[\!]{sat}}\xspace}
\newcommand{\Psat}{\ensuremath{\Pressure\subs[\!\!]{sat}}\xspace}
\newcommand{\Tsat}{\ensuremath{\Temperature\subs[\!\!\:]{sat}}\xspace}
\newcommand{\SubL}{\subs[\!\!\:]{\rule{0pt}{8pt}$\textstyle\ell$}}
\newcommand{\SubG}{\subs[\!\!\:]{$\mathit{g}$}}

\newcommand{\rhol}{\ensuremath{\rho\SubL}\xspace}
\newcommand{\rhog}{\ensuremath{\rho\SubG}\xspace}
\newcommand{\il}{\ensuremath{i\SubL}\xspace}
\newcommand{\ig}{\ensuremath{i\SubG}\xspace}
\newcommand{\rhoul}{\ensuremath{\rhou\SubL}\xspace}
\newcommand{\rhoug}{\ensuremath{\rhou\SubG}\xspace}
\newcommand{\rhoil}{\ensuremath{\rhoi\SubL}\xspace}
\newcommand{\rhoig}{\ensuremath{\rhoi\SubG}\xspace}
\newcommand{\alphal}{\ensuremath{\alpha\SubL}\xspace}
\newcommand{\alphag}{\ensuremath{\alpha\SubG}\xspace}

\newcommand{\tauSat}{\ensuremath{\tau\subs[\!\!\:]{sat}}\xspace}
\newcommand{\deltaL}{\ensuremath{\delta\subs[\!\!\:]{\rule{0pt}{8pt}$\textstyle\ell$}}\xspace}
\newcommand{\deltaG}{\ensuremath{\delta\subs[\!\!\:]{$\mathit{g}$}}\xspace}

\newcommand{\rhoc}  {\ensuremath{\rho\subs{c}}\xspace}
\newcommand{\Tc}    {\ensuremath{T\subs{c}}\xspace}

\newcommand{\Skip}[1][0.45em]{\\[#1]}
\newcommand{\TCS}    {Thermodynamic Coexistence System\xspace}
\newcommand{\TCSRef} {\hyperref[Eqn:TCS]{\TCS}\xspace}
\newcommand{\MCS}    {Mechanical Coexistence System\xspace}
\newcommand{\MCSRef} {\hyperref[Eqn:MCS]{\MCS}\xspace}

\newcommand{\Afe}{\ensuremath{A\subs{\textsc{fe}}}}
\newcommand{\HFE}{Helmholtz free energy\xspace}
\newcommand{\EOS}{equation of state\xspace}

\newcommand{\Space}{\ensuremath{z}\xspace}
\newcommand{\Time}{\ensuremath{t}\xspace}
\newcommand{\Speeds}{\ensuremath{\mathbf{\lambda}}\xspace}

\DeclareMathOperator{\Ln}{Ln}
\DeclareMathOperator{\Abs}{Abs}
\DeclareMathOperator{\Inf}{Inf}
\DeclareMathOperator{\Exp}{Exp}
\DeclareMathOperator{\Rez}{R}

\let\originalleft\left
\let\originalright\right
\renewcommand{\left}{\mathopen{}\mathclose\bgroup\originalleft\;\!}
\def\left#1{\mathopen{}\mathclose\bgroup\originalleft#1\:\!}
\def\right#1{\aftergroup\egroup\:\!\originalright#1}


%\DefineNewLength{\RowSkip}{1.0em}
%\newcommand{\skp}[1][0.45em]{
%    \ifthenelse{\equal{#1}{}}{
%        \\[\RowSkip]
%    }{
%        \\[#1]
%    }
%}

\newcommand{\Del}[1][]{
    \partial_{#1}
}

\newcommand{\Vector}[1]{
    \underline{#1}
}

\newcommand{\Tensor}[1]{
    \underline{\underline{#1}}
}

\newcommand{\qConRaw}{\mathbf{q}}
\newcommand{\qCon}{\ensuremath{\qConRaw}\xspace}
\newcommand{\qPer}{\ensuremath{\widehat{\qConRaw}}\xspace}
\newcommand{\qSS} {\ensuremath{\qConRaw^0}\xspace}

\newcommand{\ConSys}{
    \Psi
}

\newcommand{\ConSysHEM}[1][HEM]{
    \ConSys_{\!\mbox{\tiny #1}}
}


\newcommand{\Flux}{
    \mathbf{F}
}
\newcommand{\Source}{
    \mathbf{S}
}

\newcommand{\Weight}{\beta}


\newcommand{\FluxFun}[2][]{
    \mathbf{F}_{#1}\left(#2\right)
}

\newcommand{\SourceFun}[2][]{
    \mathbf{S}_{#1}\left(#2\right)
}

\newcommand{\ResidualFun}[2][]{
    \mathbf{R}_{#1}\left(#2\right)
}

\newcommand{\Jacobian}[1][]{
    \mathbb{J}\subs{#1}
}

\newcommand{\JacobGen}[2]{
  \Jacobian[{\scriptscriptstyle #1}](#2)
}

\newcommand{\JacobF}{
    \Jacobian[F]
}


\newcommand{\JacobS}[1]{
    \JacobGen{S}{#1}
}

\newcommand{\FluxSS}{
    \mathbf{F}^{0}
}

\newcommand{\SourceSS}{
    \mathbf{S}^{0}
}

\newcommand{\JacobFSS}[1][\,\,\!]{
    \mathbf{J}_{\!{\scriptscriptstyle F}}^{0}{}#1
}

\newcommand{\JacobSSS}[1][\,\,\!]{
    \mathbf{J}_{\!{\scriptscriptstyle S}}^{0}#1
}

\newcommand{\BigO}[1]{
    \ensuremath{\mathcal{O}\!\left(#1\right)}
}


\newcommand{\Correl}[2]{
    f^{\mbox{\scriptsize cor}}_{#1}\left(#2\right)
}

\newcommand{\LpNorm}[2][2]{
    \ensuremath{\lvert\!\lvert#2\rvert\!\rvert_{#1}}
}

\newcommand{\Nudge}{
    \ensuremath{\!\!\;}
}

\newcommand{\hfg}{
    \ensuremath{h_{\mbox{\scriptsize fg}}}
}



%\NewEnviron{BoxedAlgorithm}[1][H]{
%    \begin{center}
%        \begin{minipage}{0.999\textwidth}
%            \centering
%            \fcolorbox{black}{white}{
%                \centering
%                \begin{minipage}[t]{0.85\textwidth}
%                    \begin{algorithm}[#1]
%                        \BODY
%                    \end{algorithm}
%                \end{minipage}
%            }
%        \end{minipage}
%    \end{center}
%}


\DeclareRobustCommand{\TH}  {thermal hydraulics\xspace}
\DeclareRobustCommand{\THc} {Thermal hydraulics\xspace}
\DeclareRobustCommand{\THcc}{Thermal Hydraulics\xspace}
\DeclareRobustCommand{\THs} {thermal hydraulic\xspace}

\DeclareRobustCommand{\CLaw}  {conservation law\xspace}
\DeclareRobustCommand{\CLaws} {conservation laws\xspace}


\newcommand{\rhou}{\ensuremath{\rho{u}}\xspace}
\newcommand{\rhoi}{\ensuremath{\rho{i}}\xspace}

\newcommand{\tr}{\ensuremath{{}\sups{\textsc{T}}}}
\newcommand{\mdotloss}[1][]{\ensuremath{\dot{m}'''\subs[\!\!\!\!\!#1]{loss}}\xspace}
\newcommand{\Keff}{\ensuremath{K\subs{eff}}}

\newcommand{\POfRhoRhoi}{\ensuremath{P\left(\rho,\frac{\rhoi}{\rho}\right)}}


\newcommand{\EqnSkip}[1][3em]{\ensuremath{\mbox{\rule{0.5em}{#1}}}\\}
\newcommand{\psiEOS}{\ensuremath{\psi}\subs{\textsc{eos}}}




%\DefineNewLength{\BarredLetterHeight}{0pt}
%\DefineNewLength{\BarredLetterWidth}{0pt}

%\newcommand{\eBB}{
%    \ensuremath{
%        \settoheight{\BarredLetterHeight}{e} % Height in current context
%        \settowidth{\BarredLetterWidth}{e}   % Width  in current context
%        e\mbox{\hspace{-0.57\BarredLetterWidth}\rule{0.035em}{0.96\BarredLetterHeight}} % bar
%    }
%}

%\newcommand{\TableSkip}{\rule[-1.4em]{0pt}{3.3em} \\[0pt]}
\definecolor{Gray}{gray}{0.93}


\newcommand{\LedineggCriterion}{$\tfrac{\partial\Delta{P}}{\partial(\rhou)}\bigr\rvert_{\text{int}} \le 
                                 \tfrac{\partial\Delta{P}}{\partial(\rhou)}\bigr\rvert_{\text{ext}}$}
                                
                                
\newcommand{\etal}{et al.\xspace}
\newcommand{\etc}{etc.\xspace}
\newcommand{\eg}{e.g.\xspace}
\newcommand{\ie}{i.e.\xspace}


\newcommand{\rhok}{ \ensuremath{\alpha\rho\subs{\phi}}\xspace}
\newcommand{\rhouk} {\ensuremath{\alpha\rhou\subs{\phi}}\xspace}
\newcommand{\rhoik} {\ensuremath{\alpha\rhoi\subs{\phi}}\xspace}
\newcommand{\alphak}{\ensuremath{\alpha\subs{\phi}\xspace}}
\newcommand{\uk}{\ensuremath{u\subs{\phi}}\xspace}
\newcommand{\ik}{\ensuremath{i\subs{\phi}}\xspace}
\newcommand{\CVvol}[1][k]{\ensuremath{\Omega_\text{#1}}\xspace}
\newcommand{\MCvol}[1][m]{\ensuremath{\Omega_\text{#1}}\xspace}
\newcommand{\CVsurf}[1][k]{\ensuremath{\Gamma_\text{#1}}\xspace}
\newcommand{\MCsurf}[1][m]{\ensuremath{\Gamma_\text{#1}}\xspace}








    \let\Oldalpha     \alpha     \renewcommand{\alpha}     {\ensuremath{\Oldalpha     }\xspace}
    \let\Oldbeta      \beta      \renewcommand{\beta}      {\ensuremath{\Oldbeta      }\xspace}
    \let\Oldgamma     \gamma     \renewcommand{\gamma}     {\ensuremath{\Oldgamma     }\xspace}
    \let\Olddelta     \delta     \renewcommand{\delta}     {\ensuremath{\Olddelta     }\xspace}
    \let\Oldepsilon   \epsilon   \renewcommand{\epsilon}   {\ensuremath{\Oldepsilon   }\xspace}
    \let\Oldvarepsilon\varepsilon\renewcommand{\varepsilon}{\ensuremath{\Oldvarepsilon}\xspace}
    \let\Oldzeta      \zeta      \renewcommand{\zeta}      {\ensuremath{\Oldzeta      }\xspace}
    \let\Oldeta       \eta       \renewcommand{\eta}       {\ensuremath{\Oldeta       }\xspace}
    \let\Oldtheta     \theta     \renewcommand{\theta}     {\ensuremath{\Oldtheta     }\xspace}
    \let\Oldvartheta  \vartheta  \renewcommand{\vartheta}  {\ensuremath{\Oldvartheta  }\xspace}
    \let\Oldkappa     \kappa     \renewcommand{\kappa}     {\ensuremath{\Oldkappa     }\xspace}
    \let\Oldlambda    \lambda    \renewcommand{\lambda}    {\ensuremath{\Oldlambda    }\xspace}
    \let\Oldmu        \mu        \renewcommand{\mu}        {\ensuremath{\Oldmu        }\xspace}
    \let\Oldnu        \nu        \renewcommand{\nu}        {\ensuremath{\Oldnu        }\xspace}
    \let\Oldxi        \xi        \renewcommand{\xi}        {\ensuremath{\Oldxi        }\xspace}
    \let\Oldpi        \pi        \renewcommand{\pi}        {\ensuremath{\Oldpi        }\xspace}
    \let\Oldvarpi     \varpi     \renewcommand{\varpi}     {\ensuremath{\Oldvarpi     }\xspace}
    \let\Oldrho       \rho       \renewcommand{\rho}       {\ensuremath{\Oldrho       }\xspace}
    \let\Oldvarrho    \varrho    \renewcommand{\varrho}    {\ensuremath{\Oldvarrho    }\xspace}
    \let\Oldsigma     \sigma     \renewcommand{\sigma}     {\ensuremath{\Oldsigma     }\xspace}
    \let\Oldvarsigma  \varsigma  \renewcommand{\varsigma}  {\ensuremath{\Oldvarsigma  }\xspace}
    \let\Oldtau       \tau       \renewcommand{\tau}       {\ensuremath{\Oldtau       }\xspace}
    \let\Oldupsilon   \upsilon   \renewcommand{\upsilon}   {\ensuremath{\Oldupsilon   }\xspace}
    \let\Oldphi       \phi       \renewcommand{\phi}       {\ensuremath{\Oldphi       }\xspace}
    \let\Oldvarphi    \varphi    \renewcommand{\varphi}    {\ensuremath{\Oldvarphi    }\xspace}
    \let\Oldchi       \chi       \renewcommand{\chi}       {\ensuremath{\Oldchi       }\xspace}
    \let\Oldpsi       \psi       \renewcommand{\psi}       {\ensuremath{\Oldpsi}\xspace}
    \let\Oldomega     \omega     \renewcommand{\omega}     {\ensuremath{\Oldomega     }\xspace}
    \let\OldGamma     \Gamma     \renewcommand{\Gamma}     {\ensuremath{\OldGamma     }\xspace}
    \let\OldLambda    \Lambda    \renewcommand{\Lambda}    {\ensuremath{\OldLambda    }\xspace}
    \let\OldSigma     \Sigma     \renewcommand{\Sigma}     {\ensuremath{\OldSigma     }\xspace}
    \let\OldPsi       \Psi       \renewcommand{\Psi}       {\ensuremath{\OldPsi       }\xspace}
    \let\OldDelta     \Delta     \renewcommand{\Delta}     {\ensuremath{\OldDelta     }\xspace}
    \let\OldXi        \Xi        \renewcommand{\Xi}        {\ensuremath{\OldXi        }\xspace}
    \let\OldUpsilon   \Upsilon   \renewcommand{\Upsilon}   {\ensuremath{\OldUpsilon   }\xspace}
    \let\OldOmega     \Omega     \renewcommand{\Omega}     {\ensuremath{\OldOmega     }\xspace}
    \let\OldTheta     \Theta     \renewcommand{\Theta}     {\ensuremath{\OldTheta     }\xspace}
    \let\OldPi        \Pi        \renewcommand{\Pi}        {\ensuremath{\OldPi        }\xspace}
    \let\OldPhi       \Phi       \renewcommand{\Phi}       {\ensuremath{\OldPhi       }\xspace}


\UWMadSetup {
    RelativeDirectory / {
        chapter-directory-name      = same,
        chapter-directory-prefix    = Chapter-,
        section-directory-name      = none,
        subsection-directory-name   = none,
        the-only-graphics-directory = Graphics
    },
    Acronym / {
        main-title = {List of Acronyms},
        link-color = black
    }
}



% ================================================================= %
%                                Document                           %
% ================================================================= %
\begin{document}

% ----------------------------------- %
%          Title/License Pages        %
% ----------------------------------- %
    \MakeTitlePage{}
    \begin{LicensePage}
        \CreativeCommons
        \Attribution
        \NonCommercial
        \ShareAlike
    \end{LicensePage}


% ----------------------------------- %
%               Abstract              %
% ----------------------------------- %
\abstract{}
The stability of a simple, closed-loop, water-cooled natural circulation system was characterized over a range of single phase and two-phase states.
The motivation for this investigation is a Next Generation Nuclear Plant safety cooling system called the Reactor Cavity Cooling System (RCCS).
One of the proposed designs for the RCCS is a closed-circuit of network piping using water as a working fluid.  One of the safety considerations for such a system is the stability of the system at steady-state under a large number of unknown states.
This work provides a derivation of the commonly used one-dimensional conservation laws used in thermohydraulic system modeling and a novel discretization scheme that allows for exact integration of the computational domain for accurate calculation of eigenvalues of a linearized system.
The steady-state solution of the discretized equations is then performed using a fully nonlinear Jacobian-Free Newton Krylov Method for a number of temperatures, pressures, and heat loads both in single and two-phase conditions.
All of the single and two-phase state exhibit linear stability to small perturbations in values.
The linear stability is also found to increase with increasing heat load due to the greater inertia of the system damping out small perturbation effectively and with increasing pressure due to the greater stiffness of the fluid.
Nonlinear stability was also examined for a point power insertion of varying intensity from two steady-states.
The loop exhibited stability for all power insertions from both steady-states, returning to the initial steady value shortly after the pulse.



% ----------------------------------- %
%            Aknowledgements          %
% ----------------------------------- %
\acknowledgments{}
First and foremost, I would like to offer my sincerest thanks to my advisor Michael Corradini for his unending wisdom, guidance, and patience.
His encouragement and understanding provided the bedrock on which this work stands.
He has served as a paragon of knowledge, research, and achievement to which I will forevermore strive towards; a goal for which I am extremely appreciative.

I would also like to thank several colleagues for their years of camaraderie and support: Lewis, Gary, Ryan, Ryan, Billy, Mike, Brian, Amy, Megan, Tammy, Denise, and Aaron.
Additionally, I would like to offer my heartfelt appreciation for the boundless friendship of Joe, Leah, Steph, Abby, Eric, Pat, Jordan, David, Gautam, Krissy, Sam, Arrielle, Louise, Ashley, David, David, Matt, Andrew, and so many more.
Lastly, it's impossible to express how much I owe to my mother Mary, brother Bradley, and father Craig for all they have done for me as long as I can remember.









% ----------------------------------- %
%              List Pages             %
% ----------------------------------- %
    \TableOfContentsName{List of Contents}
    \TableOfContents
    \ListOfTables
    \ListOfFigures

    \begin{Acronym}
        \Entry{ANL}{Argonne National Laboratory}
        \Entry{EOS}{equation of state}
        \Entry{FDM}{Finite Difference Method}
        \Entry{FEM}{Finite Element Method}
        \Entry{FVM}{Finite Volume Method}
        \Entry{GMRES}{Generalized Minimal Residual (Method)}
        \Entry{HEM}{Homogenous Equilibrium Model}
        \Entry{IAPWS}{International Association for the Properties of Water and Steam}
        \Entry{IAPWS-95}{IAPWS Formulation 1995 for the Thermodynamic Properties of Ordinary Water Substance for General and Scientific Use}
        \Entry{MOL}{Method of Lines}
        \Entry{NGNP}{Next Generation Nuclear Plant}
        \Entry{NRC}{Nuclear Regulatory Commission}
        \Entry{ODE}{Ordinary differential equation}
        \Entry{PDE}{Partial differential equation}
        \Entry{RCCS}{Reactor Cavity Cooling System}
        \Entry{SFM}{Separated Flow Model}
    \end{Acronym}





% ----------------------------------- %
%           Chapter Includes          %
% ----------------------------------- %
    \Include{Chapter}{Introduction}
    \Include{Chapter}{Theory}
    \Include{Chapter}{Simulations}


\chapter{Conclusion}
The presented simulation results for a simple, closed loops over a number of single and two-phase states showed the system to be linearly stable under the models used and assumptions made.
The simulations were performed using the highest fidelity equations-of-state currently available for water, solved using a modern fully, nonlinear solver which avoids need to form the true Jacobian for Newton-like convergence, and used a new \Acro{FVM} discretization scheme which emphasized precise and complete coverage of the computational domain for accurate integration of the entire computational domain for the stability analysis.

Future efforts will aim to refine the state space over which the steady-states are calculated to truly exhaust potential operation zones in which the systems may become unstable to small perturbations.
Additionally, expanding the work to include larger systems in terms of both volume and path length is definitely an avenue that needs exploration.
It would be interesting to study how the character of the system changes, if at all, in scaling the small loop presented in this work to the more modest scale of the UW--Madison RCCS or even large cooling loops, such as the future experiment at Argonne National Lab.

In simulating such large systems, however, pieces of the code used to perform the simulations of this work would need to be reimplemented in a faster computer language.
The current implementation uses the MATLAB scripting language for all of the simulations and results.
MATALB was chosen for its speed of development, ease of debugging, and flexibility of programming paradigm.
Since all of the thermodynamics, solvers, etc. were written from scratch, MATLAB was used more as a prototyping language with the intent of moving to a faster language in the future.
The main hits to speed comes from the continuous \Acro{EOS} in all fluid regimes.
Massive speed increases would easily be seen if even just the two-phase checks and calculations were performed in a faster language, which is the next primary goal.

Also, the flexibility and generality of the solver is something to investigate and expand upon.
Adding a simpler model to the code base for benchmarking the full nonlinear form or even providing an initial guess would be worthwhile.
And then adding more complex models beyond HEM, such as a true two-fluid model and wall conduction/convection capabilities to the current framework would increase the usefulness of the code base and possibly provide a different picture of stability, both linear and not.


By adding more detailed models, increasing the computational efficiency of the tools, and allowing the model geometry to comes closer to reality, better investigations can be taken into the analysis of two-phase systems.
Even adding a higher level of geometric specification and mathematical rigor can lead to a more thoughtful derivation of models and consideration  of other methods to better analyze and explore these systems.





% ----------------------------------- %
%           Appendix Includes         %
% ----------------------------------- %
    \UWMadSetup{
        RelativeDirectory / {
            chapter-directory-prefix = Appendices,
            chapter-directory-name   = none,
            section-directory-name   = none
        }
    }
    \Include{Chapter}{Thermodynamics}
    \Include{Chapter}{FrictionFactor}
    \Include{Chapter}{Graphs}





% ----------------------------------- %
%           Bibliography              %
% ----------------------------------- %
    \cleardoublepage
    \setstretch{1.2}
    \bibliographystyle{References/elsarticle-num}
    \bibliography{References/Library}


\end{document}
